\documentclass[11pt,letterpaper]{article}

\usepackage[utf8]{inputenc}
\usepackage[T1]{fontenc}
\usepackage[spanish]{babel}
\usepackage{amsmath,amssymb}
\usepackage{geometry}
\usepackage{graphicx}
\usepackage{booktabs}
\usepackage{float}
\usepackage{hyperref}

\geometry{margin=1in}

\title{Taller: Validez de una Función de Autocorrelación\\Definida como Suma de Gaussianas}
\author{Martín Ramírez Espinosa\\Curso: Teoría de Señales}
\date{\today}

\begin{document}
\maketitle

\section{Planteamiento}
Se analiza la función
\begin{equation}
R(\tau)=\sum_{k\in\{-1,0,1\}}A_k\exp\left(-\frac{(\tau-\mu_k)^2}{2\sigma_k^2}\right),
\end{equation}
con
\begin{equation}
\mu_k = k,\quad A_k>0,\quad A_{-1}=A_1,\quad
\sigma_k^2=
\begin{cases}
\sigma^2, & k=0,\\
\sigma^2/|k|!, & k\neq 0.
\end{cases}
\end{equation}

Como para $k=\pm1$ se cumple $|k|!=1$, resulta $\sigma_{-1}^2=\sigma_1^2=\sigma^2$, y por tanto:
\begin{equation}
R(\tau)=A_0e^{-\tau^2/(2\sigma^2)}
+A_1e^{-(\tau-1)^2/(2\sigma^2)}
+A_1e^{-(\tau+1)^2/(2\sigma^2)}.
\end{equation}

\section{Parte Teórica}
\subsection{Propiedades básicas}
\textbf{Realidad:} cada término es real y los coeficientes son reales, luego $R(\tau)\in\mathbb{R}$.

\textbf{Simetría:}
\begin{equation}
R(-\tau)=A_0e^{-\tau^2/(2\sigma^2)}+A_1e^{-(\tau+1)^2/(2\sigma^2)}+A_1e^{-(\tau-1)^2/(2\sigma^2)}=R(\tau).
\end{equation}

\textbf{Máximo en $\tau=0$:} para una ACF válida de proceso WSS real debe cumplirse $|R(\tau)|\le R(0)$. Esta propiedad queda garantizada cuando se satisface la condición espectral de Wiener-Khinchin (sección \ref{sec:wk}).

\subsection{Positividad definida y relación con Toeplitz}
Una función es ACF válida si y solo si es positiva definida:
\begin{equation}
\sum_{m=1}^{N}\sum_{n=1}^{N} c_m c_n^*\,R(t_m-t_n)\ge 0,\quad
\forall N,\forall\{c_n\},\forall\{t_n\}.
\end{equation}

En muestreo uniforme, esto equivale a que toda matriz de Toeplitz
\begin{equation}
T_{ij}=R((i-j)\Delta t)
\end{equation}
sea semidefinida positiva.

\subsection{Teorema de Wiener-Khinchin}\label{sec:wk}
Para un proceso WSS:
\begin{equation}
S(f)=\mathcal{F}\{R(\tau)\}.
\end{equation}

La transformada de un término gaussiano desplazado es:
\begin{equation}
\mathcal{F}\left\{e^{-(\tau-\mu)^2/(2\sigma^2)}\right\}
=\sqrt{2\pi}\,\sigma\,e^{-2\pi^2\sigma^2f^2}e^{-j2\pi f\mu}.
\end{equation}

Entonces:
\begin{align}
S(f)&=\sqrt{2\pi}\,\sigma\,e^{-2\pi^2\sigma^2f^2}
\left(A_0 + A_1e^{-j2\pi f}+A_1e^{j2\pi f}\right)\\
&=\sqrt{2\pi}\,\sigma\,e^{-2\pi^2\sigma^2f^2}\left(A_0+2A_1\cos(2\pi f)\right).
\end{align}

Como el envolvente gaussiano es positivo, el signo lo determina
\begin{equation}
B(f)=A_0+2A_1\cos(2\pi f).
\end{equation}
Su minimo es $B_{\min}=A_0-2A_1$. Por tanto:
\begin{equation}
S(f)\ge 0\ \forall f
\iff A_0-2A_1\ge 0.
\end{equation}

\subsection{Conclusión teórica}
\begin{itemize}
  \item \textbf{Válida} si $A_0>2A_1$.
  \item \textbf{Caso umbral} si $A_0=2A_1$ (el espectro toca cero en frecuencias puntuales).
  \item \textbf{Inválida} si $A_0<2A_1$ (hay regiones con $S(f)<0$).
\end{itemize}
La condición de validez no depende de $\sigma$ para el signo de $S(f)$.

\section{Parte Computacional}
Se implementó el análisis en:
\begin{itemize}
  \item \texttt{notebooks/taller\_acf\_solution.py}
\end{itemize}
y se generaron:
\begin{itemize}
  \item \texttt{output/taller\_acf/metrics.json}
  \item \texttt{output/taller\_acf/summary.md}
  \item \texttt{output/taller\_acf/caso\_1\_valido.png}
  \item \texttt{output/taller\_acf/caso\_2\_umbral.png}
  \item \texttt{output/taller\_acf/caso\_3\_invalido.png}
\end{itemize}

\subsection{Configuraciones evaluadas}
\begin{enumerate}
  \item Caso 1 válido: $A_0=2.8,\ A_1=1.0,\ \sigma=0.35$.
  \item Caso 2 umbral: $A_0=2.0,\ A_1=1.0,\ \sigma=0.65$.
  \item Caso 3 inválido: $A_0=1.2,\ A_1=1.0,\ \sigma=0.90$.
\end{enumerate}

\subsection{Resultados numéricos}
\begin{table}[H]
\centering
\begin{tabular}{lrrrrr}
\toprule
Caso & $A_0-2A_1$ & $\min S(f)$ cerrada & $\min S(f)$ FFT & Negativas & $\lambda_{\min}$ Toeplitz \\
\midrule
Caso 1 válido   & 0.8  & $1.918\times10^{-7}$  & $-3.61\times10^{-16}$ & 0   & $-7.87\times10^{-14}$ \\
Caso 2 umbral   & 0.0  & $0.0$                 & $-1.02\times10^{-14}$ & 0   & $-7.16\times10^{-14}$ \\
Caso 3 inválido & -0.8 & $-7.51\times10^{-2}$  & $-7.48\times10^{-2}$  & 946 & $-1.31\times10^{-9}$ \\
\bottomrule
\end{tabular}
\caption{Validación numérica de la condición teórica de validez.}
\end{table}

Los valores negativos de orden $10^{-14}$ en casos válidos son error numérico. En el caso inválido, la negatividad es macroscópica, por lo que no puede atribuirse a redondeo.

\subsection{Gráficas}
\begin{figure}[H]
  \centering
  \includegraphics[width=\textwidth]{output/taller_acf/caso_1_valido.png}
  \caption{Caso 1 válido: espectro no negativo.}
\end{figure}

\begin{figure}[H]
  \centering
  \includegraphics[width=\textwidth]{output/taller_acf/caso_2_umbral.png}
  \caption{Caso 2 umbral: el espectro toca cero.}
\end{figure}

\begin{figure}[H]
  \centering
  \includegraphics[width=\textwidth]{output/taller_acf/caso_3_invalido.png}
  \caption{Caso 3 inválido: regiones negativas en $S(f)$.}
\end{figure}

\section{Respuestas a las preguntas de análisis}
\begin{enumerate}
  \item \textbf{No} toda suma de gaussianas define una ACF válida: debe cumplir simetría y no negatividad espectral.
  \item Las gaussianas desplazadas introducen la modulación cosenoidal $2A_1\cos(2\pi f)$ que puede forzar negatividad.
  \item Sí: el umbral exacto es $A_0=2A_1$.
  \item Si es válida, describe un proceso WSS con covarianza central más dos lóbulos laterales simétricos.
  \item Si el espectro es negativo en alguna frecuencia, la función no puede ser PSD física de un proceso WSS real.
\end{enumerate}

\section{Conclusión final}
La función propuesta \textbf{no es siempre válida}.  
Es \textbf{válida bajo ciertas condiciones}, exactamente cuando:
\begin{equation}
A_0\ge 2A_1.
\end{equation}
Si $A_0<2A_1$, \textbf{no es válida en general} por violar $S(f)\ge 0$.

\end{document}
